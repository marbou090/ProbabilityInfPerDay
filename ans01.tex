\documentclass[a4j,uplatex]{jsarticle}
\usepackage{mathrsfs}
\usepackage{amsmath}

\title{確率情報理論第1回 解答}
\author{加藤まる}
\date{2020/03/01}

\begin{document}
\maketitle
\bf キーワード:確率空間,確率分布,分布関数,幾何分布,ファーストサクセス分布
\rm

\section*{本日の問題解答}
1の目がでる確率が $\frac{1}{2}$ で、他の目については均等な確率であるインチキなサイコロを考える。
\begin{itemize}
  \item[(1)] このサイコロを1個投げるときの確率空間を求めよ。 
  \\ 
  標本空間$\Omega $は、
  \begin{equation}
    \Omega = \{ 1,2,3,4,5,6 \} 
  \end{equation}
  である。次に事象の集合 $\mathscr F$は、
  \begin{equation}
    \begin{split}
       \mathscr F &= \{ \emptyset,\{ 0 \},\{ 1\},\{ 2\},\{ 3\},\{ 4\},\{ 5\},\{ 6\},\\
      &\{ 1,2\},\{ 1,3\},\{ 1,4\},\{ 1,5\},\{ 1,6\},\{ 2,3\},\{ 2,4\},\{ 2,5\},\{ 2,6\},\\
      &\{ 3,4\},\{ 3,5\},\{ 3,6\},\{ 4,5\},\{ 4,6\},\{ 5,6\},\\
      &\{ 1,2,3\},\{ 1,2,3\},\{ 1,2,4\},\{ 1,2,5\},\{ 1,2,6\},\{ 1,3,4\},\{ 1,3,5\},\{ 1,3,6\},\\
      &\{ 1,4,5\},\{ 1,4,6\},\{ 1,5,6\},\\
      &\{ 2,3,4\},\{ 2,3,5\},\{ 2,3,6\},\{ 2,4,5\},\{ 2,4,6\},\{ 2,5,6\},\\
      &\{ 3,4,5\},\{ 3,4,6\},\{ 3,5,6\},\{ 4,5,6\},\\
      &\{ 1,2,3,4\},\{ 1,2,3,5\},\{ 1,2,3,6\},\{ 1,2,4,5\},\{ 1,2,4,6\},\{ 1,3,4,5\},\{ 1,3,4,6\},\\
      &\{ 1,4,5,6\},\{ 2,3,4,5\},\{ 2,3,4,6\},\{ 2,3,5,6\},\{ 2,4,5,6\},\{ 3,4,5,6\},\\
      &\{ 1,2,3,4,5\},\{ 1,2,3,4,6\},\{ 1,2,3,5,6\},\{ 1,2,4,5,6\},\{ 1,3,4,5,6\},\{ 2,3,4,5,6\} \}\\
    \end{split}
  \end{equation}
  である。最後に確率測度Pは、
  \begin{equation}
    P(\{1\})=\frac{1}{2},P(\{2\})= \cdots = P(\{6\}) = \frac{1}{10}
  \end{equation}

  \item[(2)] このサイコロを1個投げたとき、偶数が出る確率を求めよ。
  \begin{equation}
    P(\{2,4,6\})=\frac{3}{10}
  \end{equation} 
  \item[(3)] このサイコロを2個投げたとき、2つとも奇数の目が出る確率を求めよ。
    \\
    まずこのサイコロ1個を投げたとき、奇数の目が出る確率は、
  \begin{equation}
    \begin{split}
      P(\{1,3,5\})&=\frac{1}{2} + \frac{1}{10} + \frac{1}{10}\\
      &=\frac{7}{10}
    \end{split}
  \end{equation}  
  である。2つの同時確率を考えるので、
  \begin{equation}
    \begin{split}
      \frac{7}{10}\times \frac{7}{10} = \frac{49}{100}
    \end{split}
  \end{equation}  
  である。
\end{itemize}

\section*{おかわり問題解答}
\begin{itemize}
  \item[(1)] Xの確率分布$(P=X)$、Yの確率分布を求めよ。
  \item[(2)] Xの分布関数を求めよ。 
  \item[(3)] $ P(X\ge20) , P(X\ge Y<30) $を求めよ。 
\end{itemize}

\section*{解説}

\end{document}