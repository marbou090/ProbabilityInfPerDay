\documentclass[a4j,uplatex]{jsarticle}

\title{確率情報理論第1回}
\author{加藤まる}
\date{2020/03/01}

\begin{document}
\maketitle
\section*{本日の問題}

1の目がでる確率が $\frac{1}{2}$ で、他の目については均等な確率であるインチキなサイコロを考える。
\begin{itemize}
  \item[(1)] このサイコロを1個投げるときの確率空間を求めよ。 
  \item[(2)] このサイコロを1個投げたとき、偶数が出る確率を求めよ。
  \item[(3)] このサイコロを2個投げたとき、奇数の目が出る確率を求めよ。 
\end{itemize}

\section*{おかわり問題}
普通のサイコロを何回も投げる。初めて6がでるまでに6以外が出た回数をXとする。また、初めてY回目に6が出たとする。
\begin{itemize}
  \item[(1)] Xの確率分布$(P=X)$、Yの確率分布を求めよ。
  \item[(2)] Xの分布関数を求めよ。 
  \item[(3)] $ P(X\ge20) , P(X\ge Y<30) $を求めよ。 
\end{itemize}

解答を加藤まる(まるぼう)にDiscordDMに送ると添削します(添削不要の場合DMは不要)。解答は夜にDiscordに貼るので自己採点してみてください。

\end{document}