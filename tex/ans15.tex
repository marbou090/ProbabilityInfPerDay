\documentclass[a4j,uplatex,dvipdfmx]{jsarticle}
\usepackage{amsmath}
\usepackage{tcolorbox}
\tcbuselibrary{breakable, skins, theorems}

\title{確率情報理論第15回 解答}
\author{加藤まる}
\date{2020/03/15}

\begin{document}
\maketitle
\bf キーワード:
\rm

\section*{本日の問題解答}
$f_x(x)=2\theta x e^{-\theta x^2}~~(x>0)$とし、$X$に従う母集団から$n$個の標本$x_1,x_2,\cdots , x_n$をとる。
母パラメータ$\theta$の最尤推定値$\hat{\theta}$を求めよ。

尤度関数は、
\begin{equation}
  L(x_1,x_2,\cdots, x_n|\theta)=2^n\theta^n x_1  x_2\cdots x_n e^{-\theta(x_1^2+\cdots+x_n^2)}
\end{equation}
となることから、
\begin{equation}
    \log{L}= n\log{2}+n\log{\theta} + n\log(x_1x_2\cdots x_n)-\theta(x_1^2+x_2^2+\cdots+x_n^2)
\end{equation}
\begin{equation}
  0=\frac{\partial \log{L}}{\partial \theta}=\frac{n}{\theta}-(x_1^2+x_2^2+\cdots+x_n^2)
\end{equation}
よって、$\displaystyle \hat{\theta}=\frac{n}{x_1^2+x_2^2+\cdots+x_n^2}$である。


\section*{おかわり問題解答}
$Exp(\frac{1}{\lambda})$母集団(平均$\lambda$)での母平均$\lambda$の最尤推定値$\hat{\lambda}$を求めよ。
\\ \\ 
尤度関数は
\begin{equation}
  L(x_1,x_2,\cdots, x_n|\lambda) = \lambda^{-n}e^{-\frac{x_1,x_2,\cdots, x_n}{\lambda}}
\end{equation}
よって、$\log{L}=-n\log{\lambda}-\lambda^{-1}(x_1,x_2,\cdots, x_n)$である。つまり、
\begin{equation}
  \begin{split}
    0=\frac{\partial \log{L}}{\partial \lambda}&=\frac{\partial}{\partial \lambda}(-n\log{\lambda}-\lambda^{-1}(x_1,x_2,\cdots, x_n)) \\
    &=-\frac{n}{\lambda}+\frac{1}{\lambda^2}(x_1,x_2,\cdots, x_n)
  \end{split}
\end{equation}
よって、求める最尤推定値$\hat{\lambda}$は、
\begin{equation}
  \hat{\lambda}=\frac{x_1,x_2,\cdots, x_n}{n}=\bar{x}
\end{equation}


\end{document}