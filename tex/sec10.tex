\documentclass[a4j,uplatex,dvipdfmx]{jsarticle}
\usepackage{amsmath}
\usepackage{tcolorbox}
\tcbuselibrary{breakable, skins, theorems}

\title{確率情報理論第10回}
\author{加藤まる}
\date{2020/03/10}

\begin{document}
\maketitle

\section*{本日の問題}
標本空間$A={0,1}$として、$A$上の2つの確率分布$PとQ$を
\begin{center}
  $P(0)=P(1)=0.5,~~Q(0)=0.2,~~Q(1)==0.8$
\end{center}
と定める。このとき、$D(P||Q)\neq D(Q||P)$を示せ。


\section*{おかわり問題}
定数$O<a,b<1$が与えられたとする。集合$\{ 0,1,2,\cdots ,n \}$上の2つの2項分布
\begin{center}
  $p(r) = {}_nC_r~a^r (1-a)^{n-r}$\\
  $q(r) = {}_nC_r~b^r (1-b)^{n-r}$\\
\end{center}
のカルバック距離を$D(p||q)$とすると、$a,b$の関数$f(a,b)$が存在して、
\begin{center}
  $D(p||q)=nf(a,b)$
\end{center}
とかけることを示せ。また、$f(a,b)$を求めよ。\\
\\
解答を加藤まる(まるぼう)にDiscordDMに送ると添削します(添削不要の場合DMは不要)。解答は夜にDiscordに貼るので自己採点してみてください。

\end{document}