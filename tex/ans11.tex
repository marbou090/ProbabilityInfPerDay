\documentclass[a4j,uplatex,dvipdfmx]{jsarticle}
\usepackage{amsmath}
\usepackage{tcolorbox}
\tcbuselibrary{breakable, skins, theorems}

\title{確率情報理論第11回 解答}
\author{加藤まる}
\date{2020/03/11}

\begin{document}
\maketitle
\bf キーワード:
\rm

\section*{本日の問題解答}
母集団分布$F$の分散$\sigma ^2$が$5.0^2$であるものとして、この母集団からとった大きさ$n=100$の
標本の標本平均$\bar{x}=122.0$から、母平均$\mu$の$95\%$信頼区間を求めよ。
\\ \\ 
$z_{0.025}=1.960$であることより、
\begin{equation}
  122.0-1.960\times \frac{5.0}{\sqrt{100}} < \mu < 122.0+1.960\times \frac{5.0}{\sqrt{100}}
\end{equation}
よって、信頼区間は$121.0<\mu < 123.0$である。


\section*{おかわり問題解答}
二項分布$B(1,p)$を母集団分布にもつある母集団からとった大きさ100の標本の標本平均が、
$\bar{x}=0.61$であった。このとき、割合$p$の$95 \%$信頼区間を求めよ。($z_α =1.960$)\\
割合pに対する$100(1-α)\%$信頼区間は、
\begin{equation}
  \bar{x}-\frac{z_{\frac{α}{2}}}{2\sqrt{n}} < p < \bar{x}+\frac{z_{\frac{α}{2}}}{2\sqrt{n}}
\end{equation}
で求められることから、
\begin{equation}
  0.61 - \frac{1.960}{2\sqrt{100}} < p < 0.61 + \frac{1.960}{2\sqrt{100}}
\end{equation}
よって、信頼区間は$0.512<p<0.708$である。


\section*{おかわり問題解説}

\end{document}