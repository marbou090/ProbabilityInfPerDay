\documentclass[a4j,uplatex]{jsarticle}
\usepackage{amsmath}

\title{確率情報理論第5回}
\author{加藤まる}
\date{2020/03/05}

\begin{document}
\maketitle

\section*{本日の問題}
以下の場合に$\displaystyle \lim_{n \to \infty} \frac{X_1 + X_2 + \cdots +X_n}{n}$と$\displaystyle \lim_{n \to \infty} \frac{X_1 + X_2 + \cdots +X_n}{X_1^2 + X_2^2 + \cdots +X_n^2}$を大数の法則を用いて求めよ。
\begin{itemize}
  \item[(1)] $X_1\sim X_2\sim \cdots \sim B(n,p)$
  \item[(2)] $X_1\sim X_2\sim \cdots \sim N(n,\sigma ^2)$
\end{itemize}
また、$X_1\sim X_2\sim \cdots \sim B(n,p)$は確率変数$X_1, X_2,\cdots$が二項分布に近似できるという意であり、(2)も正規分布に近似できるという意である。


\section*{おかわり問題}
\bf しっぽの定理\rm より、
\begin{equation}
  E(T)=\int_{0}^{\infty} P(T\ge x)dx
\end{equation}
である。($F_T (x)$は分布関数である。)また、分布関数の定義より
\begin{equation}
  P(X\ge x) = \int_{0}^{\infty} f(t)dt
\end{equation}
である。このことを用いて、$E[Exp(\lambda)]$を求めよ。($Exp(\lambda)=\lambda e^{-\lambda t}$である。)
\\
ヒント:二重積分が出てきます。
\begin{equation}
  \begin{split}
    E[Exp(\lambda)] &= \int_{0}^{\infty} P(Exp(\lambda)\ge x)dx \\
    &=\int_{0}^{\infty} \int_{x}^{\infty} Exp(\lambda)~ dt dx
  \end{split}
\end{equation}
\\
\\
\\
解答を加藤まる(まるぼう)にDiscordDMに送ると添削します(添削不要の場合DMは不要)。解答は夜にDiscordに貼るので自己採点してみてください。

\end{document}