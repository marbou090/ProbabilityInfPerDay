\documentclass[a4j,uplatex,dvipdfmx]{jsarticle}
\usepackage{amsmath}
\usepackage{tcolorbox}
\tcbuselibrary{breakable, skins, theorems}

\title{確率情報理論第10回 解答}
\author{加藤まる}
\date{2020/03/10}

\begin{document}
\maketitle
\bf キーワード:
\rm

\section*{本日の問題解答}
標本空間$A={0,1}$として、$A$上の2つの確率分布$PとQ$を
\begin{center}
  $P(0)=P(1)=0.5,~~Q(0)=0.2,~~Q(1)=0.8$
\end{center}
と定める。このとき、$D(P||Q)\neq D(Q||P)$を示せ。
\\ \\ 
\begin{equation}
  D(P||Q) = 0.5\log{\frac{0.5}{0.2}}+0.5\log{\frac{0.5}{0.8}} \approx 0.3219
\end{equation}
\begin{equation}
  D(Q||P)=0.2\log{\frac{0.2}{0.5}} +0.8\log{\frac{0.8}{0.5}}\approx 0.2781
\end{equation}
よって、$D(P||Q)\neq D(Q||P)$が示せた。


\section*{おかわり問題解答}
定数$0<a,b<1$が与えられたとする。集合$\{ 0,1,2,\cdots ,n \}$上の2つの二項分布
\begin{center}
  $p(r) = {}_nC_r~a^r (1-a)^{n-r}$\\
  $q(r) = {}_nC_r~b^r (1-b)^{n-r}$\\
\end{center}
のカルバック情報量を$D(p||q)$とすると、$a,b$の関数$f(a,b)$が存在して、
\begin{center}
  $D(p||q)=nf(a,b)$
\end{center}
とかけることを示せ。また、$f(a,b)$を求めよ。\\
\begin{equation}
  \begin{split}
    D(p||q) &= \sum_{k=0}^n {}_nC_k a^k (1-a)^{n-k} \log{\frac{{}_nC_k a^k (1-a)^{n-k}}{{}_nC_k b^k (1-b)^{n-k}}} \\ 
    &=\sum_{k=0}^n {}_nC_k a^k (1-a)^{n-k} \log{\frac{ a^k (1-a)^{n-k}}{ b^k (1-b)^{n-k}}} \\
    &= \sum_{k=0}^n {}_nC_k a^k (1-a)^{n-k} \log \left( {\left(\frac{a}{b} \right)^k \left(\frac{1-a}{1-b}\right)^{n-k}} \right) \\ 
    &=\sum_{k=0}^n {}_nC_k a^k (1-a)^{n-k} \left(k \log \left( \frac{a}{b} \right) + (n-k)\log \left( \frac{1-a}{1-b} \right) \right)
  \end{split}
\end{equation}
分けて考えていく。
\begin{equation}
    \sum_{k=0}^n {}_nC_k a^k (1-a)^{n-k} k \log \left( \frac{a}{b} \right) = \log \left( \frac{a}{b} \right) \sum_{k=0}^n k~{}_nC_k a^k (1-a)^{n-k} \\
\end{equation}
このとき、右辺は二項分布の期待値になっていることから、
\begin{equation}
  \sum_{k=0}^n {}_nC_k a^k (1-a)^{n-k} k \log \left( \frac{a}{b} \right) = na\log \left( \frac{a}{b} \right)
\end{equation}
である。次に、
\begin{equation}
  \sum_{k=0}^n {}_nC_k a^k (1-a)^{n-k}(n-k)\log \left( \frac{1-a}{1-b} \right) =\log \left( \frac{1-a}{1-b} \right) \sum_{k=0}^n {}_nC_k a^k (1-a)^{n-k}(n-k)
\end{equation}
である。右辺について分解していくと、
\begin{equation}
  \log \left( \frac{1-a}{1-b} \right) \left( \sum_{k=0}^n n~{}_nC_k a^k (1-a)^{n-k} - \sum_{k=0}^n k~{}_nC_k a^k (1-a)^{n-k} \right)
\end{equation}
である。二項定理より、
\begin{equation}
  (a+b)^n = \sum_{k=0}^n {}_nC_k a^n b^{n-k}
\end{equation}
であるから、左側の部分は$b=(1-a)$と見て1になる。右側の部分は二項分布の期待値より$na$になっている。\\
以上より、
\begin{equation}
    na\log{\frac{a}{b}} + \log{\frac{1-a}{1-b}}(n-na) = n \left( a\log{\frac{a}{b}} + (1-a)\log{\frac{1-a}{1-b}} \right)
\end{equation}
よって、
\begin{equation}
  f(a,b)= \left( a\log{\frac{a}{b}} + (1-a)\log{\frac{1-a}{1-b}} \right)
\end{equation}

\end{document}