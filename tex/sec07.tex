\documentclass[a4j,uplatex]{jsarticle}
\usepackage{amsmath}

\title{確率情報理論第7回}
\author{加藤まる}
\date{2020/03/07}

\begin{document}
\maketitle

\section*{本日の問題}
確率変数$X$が正規分布$N(0,1)$に従うとき、確率$P(|X|\ge2)$について。
\begin{itemize}
  \item[(1)] 標準正規分布表から真の値を求めよ
  \item[(2)] チェビシェフの不等式による上からの評価値と比較せよ。(値はかなり異なる) 
\end{itemize}


\section*{おかわり問題}
確率変数$X$が
\begin{equation}
  f(x)=
 \begin{cases}
   x+1~~~(-1\le x<0)\\
   1-x~~~(0\le x<1)\\
   0~~~~~(otherwise)\\
 \end{cases}
\end{equation}
に従う場合を考える。
\begin{itemize}
  \item[(1)] 期待値を求めよ。
  \item[(2)] 分散をも求めよ。
  \item[(3)] $P(|X|>\frac{1}{4})$となる確率をチェビシェフの不等式から求めよ。\\
\end{itemize}
解答を加藤まる(まるぼう)にDiscordDMに送ると添削します(添削不要の場合DMは不要)。解答は夜にDiscordに貼るので自己採点してみてください。
\end{document}