\documentclass[a4j,uplatex,dvipdfmx]{jsarticle}
\usepackage{amsmath}
\usepackage{tcolorbox}
\tcbuselibrary{breakable, skins, theorems}

\title{確率情報理論第16回}
\author{加藤まる}
\date{2020/03/16}

\begin{document}
\maketitle

\section*{本日の問題}
$U(0,\theta)$に従う母集団での母パラメータ$\theta$の不偏推定量$T=2X,T'=\frac{n+1}{n}~max(X_1,X_2,\cdots,X_n)$
のどちらがより有効であるか調べよ。なお、分散が小さいほどより有効であるとする。


\section*{おかわり問題}
$Exp(\frac{1}{\lambda})$に従う母集団において、母平均$\lambda$の不偏推定量$\bar{X}$と
$T=c_n~min(X_1,X_2,\cdots,X_n)$を考える。
\begin{itemize}
  \item[(1)]$c_n$を求めよ。
  \item[(2)]$\bar{X}$と$T$のどちらがより有効か調べよ。
  \item[(3)]$\bar{X}$のフィッシャー情報量を求め、クラメール=ラオの不等式の下限
  と一致することを確かめ、有効推定量であることを示せ。
\end{itemize}



解答を加藤まる(まるぼう)にDiscordDMに送ると添削します(添削不要の場合DMは不要)。解答は夜にDiscordに貼るので自己採点してみてください。

\end{document}