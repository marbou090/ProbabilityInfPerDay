\documentclass[a4j,uplatex]{jsarticle}

\title{確率情報理論第4回}
\author{加藤まる}
\date{2020/03/04}

\begin{document}
\maketitle

\section*{本日の問題}
以下の場合に$\displaystyle \lim_{n \to \infty} \frac{X_1 + X_2 + \cdots +X_n}{n}$と$\displaystyle \lim_{n \to \infty} \frac{X_1 + X_2 + \cdots +X_n}{X_1^2 + X_2^2 + \cdots +X_n^2}$を大数の法則を用いて求めよ。
\begin{itemize}
  \item[(1)] $X_1\sim X_2\sim \cdots \sim B(n,p)$
  \item[(2)] $X_1\sim X_2\sim \cdots \sim N(n,\sigma ^2)$
\end{itemize}
また、$X_1\sim X_2\sim \cdots \sim B(n,p)$は確率変数$X_1, X_2,\cdots$が二項分布に近似できるという意であり、(2)も正規分布に近似できるという意である。


\section*{おかわり問題}
\bf しっぽの定理\rm より、
\begin{equation}
  E(T)=\int_{0}^{\infty} P(T\ge x)dx = \int_{0}^{\infty} (1-F_T (x))dx
\end{equation}
である。($F_T (x)$は分布関数である。)また、分布関数の定義より
\begin{equation}
  F_T(x) = P_t(X\le x) = \int_{-\infty}^{x} f(t)dx
\end{equation}
である。このことを用いて以下を求めよ。
\\
\\
\\
解答を加藤まる(まるぼう)にDiscordDMに送ると添削します(添削不要の場合DMは不要)。解答は夜にDiscordに貼るので自己採点してみてください。

\end{document}