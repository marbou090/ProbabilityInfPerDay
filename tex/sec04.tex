\documentclass[a4j,uplatex]{jsarticle}

\title{確率情報理論第4回}
\author{加藤まる}
\date{2020/03/04}

\begin{document}
\maketitle

\section*{本日の問題}
確率変数$X$が標準正規分布に従うとき、以下を求めよ。
\begin{itemize}
  \item[(1)] $f_X(x)$
  \item[(2)] $E[X]$
  \item[(3)] $V[X]$
\end{itemize}


\section*{おかわり問題}
確率変数$A,B$がそれぞれ$A\sim N(0,1),~B\sim (N2,5)$となっている。$\Phi (x)=P(N(0,1)\ge x)$は標準正規分布の分布関数である。
このことを用いて以下を証明せよ。
\begin{itemize}
  \item[(1)] $E[A^{99}]$ 
  \item[(2)] $F_B(x)$
\end{itemize}



解答を加藤まる(まるぼう)にDiscordDMに送ると添削します(添削不要の場合DMは不要)。解答は夜にDiscordに貼るので自己採点してみてください。

\end{document}