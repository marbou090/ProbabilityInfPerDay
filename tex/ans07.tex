\documentclass[a4j,uplatex]{jsarticle}

\title{確率情報理論第7回 解答}
\author{加藤まる}
\date{2020/03/07}

\begin{document}
\maketitle
\bf キーワード:
\rm

\section*{本日の問題解答}
確率変数$X$が正規分布$N(0,1)$に従うとき、確率$P(|X|\ge2)$について。
\begin{itemize}
  \item[(1)] 標準正規分布表から真の値を求めよ\\
    標準正規分布表より、$x=2.0$では0.0228であるから、$0.0228\times 2=0.0456$が真の値である。
  \item[(2)] チェビシェフの不等式による上からの評価値と比較せよ。(値はかなり異なる) \\
    
\end{itemize}


\section*{本日の問題解説}

\section*{おかわり問題解答}
確率変数$X$が
\begin{equation}
  f(x)=
 \begin{cases}
   x+1~~~(-1\le x<0)\\
   1-x~~~(0\le x<1)\\
   0~~~~~(otherwise)\\
 \end{cases}
\end{equation}
に従う場合を考える。
\begin{itemize}
  \item[(1)] 期待値を求めよ。
  \item[(2)] 分散をも求めよ。
  \item[(3)] $P(|X|>\frac{1}{4})$となる確率をチェビシェフの不等式から求めよ。\\
\end{itemize}

\section*{おかわり問題解説}

\end{document}