\documentclass[a4j,uplatex,dvipdfmx]{jsarticle}
\usepackage{amsmath}
\usepackage{tcolorbox}
\usepackage{amssymb}
\tcbuselibrary{breakable, skins, theorems}

\title{確率情報理論第12回 解答}
\author{加藤まる}
\date{2020/03/12}

\begin{document}
\maketitle
\bf キーワード:
\rm

\section*{本日の問題解答}
正規母集団$N(\mu ,\sigma ^2)$からのn個の標本$X_1,X_2,\cdots ,X_n$とその標本平均$\bar{X}$、不偏標本分散$\hat{S}^2$について、次を求めよ。
\begin{itemize}
  \item[(1)] $E[\bar{X}]$
  期待値の線形性から
  \begin{equation}
    E(\bar{X})=E\left( \frac{X_1+X_2+\cdots +X_n}{n} \right) = \frac{n\mu}{n}=\mu
  \end{equation} 
  \item[(2)] $V[\bar{X}]$
  \begin{equation}
    V(\bar{X})=V\left( \frac{X_1+X_2+\cdots+X_n}{n} \right) = \frac{n\sigma ^2}{n^2}=\frac{\sigma ^2}{n}
  \end{equation} 
  \item[(3)] $\bar{X}$の分布\\
  (1),(2)と正規分布の再現性より、$\bar{X}\sim N(\mu,\sigma ^2/n)$
  \item[(4)] $E[\hat{S}^2]$   
  \begin{equation}
    \begin{split}
      E(\hat{S}^2) &=\frac{1}{(n-1)}E\left( \sum_{i=1}^n (X_i-\bar{X})^2 \right) \\ 
      &= \frac{1}{n-1}E\left( \sum_{i=1}^n (X_i^2 - 2X_i\bar{X} + \bar{X}^2) \right) \\
      &= \frac{1}{n-1}(nE(X_1^2)-2nE(\bar{X}^2)+nE(\bar{X}^2))\\
      &= \frac{1}{n-1}(n(V(X_1)+E(X_1)^2)-n(V(\bar{X})+E(\bar{X})^2))\\
      &= \frac{n}{n-1}\left( \sigma ^2 - \frac{\sigma ^2}{n} \right) = \sigma ^2
    \end{split}
  \end{equation} 
\end{itemize}

\section*{おかわり問題解答}
平均$\mu$,分散$\sigma ^2$の母集団からとった2個の無作為標本を$X_1,X_2$とす
るとき,$aX_1+bX_2$が$\mu$の不偏推定量で,かつ有効推定量となるような$a,b$の値を求めよ。\\
\\
$X_1,X_2$は無作為標本だから、
\begin{equation}
  E(X_1)=E(X_2)=\mu,~~~V(X_1)=V(X_2)=\sigma \\
\end{equation}
$aX_1+bX_2$の期待値は
\begin{equation}
  E(aX_1+bX_2)=\mu ~~~~ \therefore a+b=1 
\end{equation}
\begin{equation}
  V(aX_1+bX_2)=a^2 V(X_1)+b^2V(X_2)
\end{equation}
$X_1,X_2$は独立だから、
\begin{equation}
  V(aX_1+bX_2)=(a^2+b^2)\sigma ^2
\end{equation}
$V(aX_1+bX_2)$を最小にするa,bの値は
\begin{equation}
  \begin{split}
    a^2+b^2&=a^2+(1-a)^2\\
    &=2a^2-2a+1\\
    &=2\left( a-\frac{1}{2} \right)^2 + \frac{1}{2}
  \end{split}
\end{equation}
より、$a=\frac{1}{2},b=1-a=\frac{1}{2}$である。\\
よって、標本平均$\frac{X_1+X_2}{2}$は$\mu$の不偏で、かつ有効な推定量である。


\end{document}