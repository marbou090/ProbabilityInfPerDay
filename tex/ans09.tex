\documentclass[a4j,uplatex,dvipdfmx]{jsarticle}
\usepackage{amsmath}
\usepackage{tcolorbox}
\tcbuselibrary{breakable, skins, theorems}

\title{確率情報理論第9回 解答}
\author{加藤まる}
\date{2020/03/09}

\begin{document}
\maketitle
\bf キーワード:
\rm

\section*{本日の問題解答}
箱に$N$個の球が入っている。$m$個は白球、$N-m$が黒球である。ここから$n$個の球を抜き出すとき、
白球の個数を$X$個とする。白球に$1 \sim m$の番号をつけたとする。
\begin{equation}
  Y_i=\\
  \begin{cases}
    1~~~i番目の白球が選ばれた場合\\
    0~~~i番目の白球が選ばれなかった場合\\
  \end{cases}
\end{equation}
としたとき、$X$を$Y_i$で表し、$E[Y_i],V[Y_i]$を求め、$E[X],V[X]$を求めよ。\\
\\
まず、$X=Y_1+Y_2+\cdots +Y_m$である。$E[Y_i]=\frac{n}{N}$より、
\begin{equation}
  E[X] = E(Y_1+Y_2+\cdots +Y_m) = \frac{mn}{N}
\end{equation}
また、
\begin{equation}
  V[Y_i]=\frac{n}{N}\left( 1-\frac{n}{N} \right)
\end{equation}
となる。($Y_i$~$Be\left( \frac{n}{N} \right)$)


\end{document}