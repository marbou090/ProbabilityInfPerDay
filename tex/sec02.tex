\documentclass[a4j,uplatex]{jsarticle}

\title{確率情報理論第2回}
\author{加藤まる}
\date{2020/03/02}

\begin{document}
\maketitle

\section*{本日の問題}
1〜5の数字が書かれたカードがある。それぞれのカードを引く確率が、書かれた数字に比例する場合を考える。$(\Omega=\{ 1,2,3,4,5\})$
\begin{itemize}
  \item[(1)] 確率関数$P(X=k)$を求めよ。
  \item[(2)] $E[X]$を求めよ。
  \item[(3)] 確率変数$X(\omega)=\omega$の分布関数を求めて図示せよ。
\end{itemize}

\section*{おかわり問題}
裏と表が描かれてるコインがある。表の出る事象を1、裏の出る事象を0とする$(確率変数Y=\{ 0,1\})$。このとき、上の問題の確率変数$X$との同時確率を考えていく。
\begin{itemize}
  \item[(1)] $X$と$Y$の同時確率を求めよ。
  \item[(2)] 条件付き期待値$E[X=k|Y=1]$を求めよ。
\end{itemize}

解答を加藤まる(まるぼう)にDiscordDMに送ると添削します(添削不要の場合DMは不要)。解答は夜にDiscordに貼るので自己採点してみてください。

\end{document}