\documentclass[a4j,uplatex,dvipdfmx]{jsarticle}
\usepackage{tcolorbox}
\newtheorem{theorem1}{定理}
\tcbuselibrary{breakable, skins, theorems}

\title{確率情報理論第8回}
\author{加藤まる}
\date{2020/03/08}

\begin{document}
\maketitle

\section*{本日の問題}
確率変数$X$が指数分布$Ex(1)$に従うとき、確率$P(X\le 2)$について、真の値とマルコフの不等式による上からの評価値を比較せよ。(値はかなり異なる)

\section*{おかわり問題}
イェンセンの不等式は大数の弱法則の導出などに用います。
  \begin{tcolorbox}[
  title = イェンセンの不等式,
  ]
    \begin{theorem1}
      $g$が2回微分可能で下に凸($g''(x)\ge 0$)であるなら、$g(E[X])\le E(g(X))$
    \end{theorem1}
  \end{tcolorbox}
$a_1,a_2,\cdots ,a_n \ge 0$として、イェンセンの不等式を用いて$\frac{a_1+a_2+\cdots +a_n}{n}\ge \sqrt[n]{a_1a_2\cdots a_3}$\\を示せ。
\\
\\
解答を加藤まる(まるぼう)にDiscordDMに送ると添削します(添削不要の場合DMは不要)。解答は夜にDiscordに貼るので自己採点してみてください。

\end{document}