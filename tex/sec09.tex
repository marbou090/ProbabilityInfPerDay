\documentclass[a4j,uplatex]{jsarticle}
\usepackage{amsmath}
\usepackage{tcolorbox}
\tcbuselibrary{breakable, skins, theorems}

\title{確率情報理論第9回}
\author{加藤まる}
\date{2020/03/09}

\begin{document}
\maketitle

\section*{本日の問題}
箱に$N$個の球が入っている。$m$個は白球、$N-m$が黒球である。ここから$n$個の球を抜き出すとき、
白球の個数を$X$個とする。白球に$1 \sim m$の番号をつけたとする。
\begin{equation}
  Y_i=\\
  \begin{cases}
    1~~~i番目の白球が選ばれた場合\\
    0~~~i番目の白球が選ばれなかった場合\\
  \end{cases}
\end{equation}
としたとき、$X$を$Y_i$で表し、$E[Y_i],V[Y_i]$を求め、$E[X],V[X]$を求めよ。\\
\\
解答を加藤まる(まるぼう)にDiscordDMに送ると添削します(添削不要の場合DMは不要)。解答は夜にDiscordに貼るので自己採点してみてください。

\end{document}