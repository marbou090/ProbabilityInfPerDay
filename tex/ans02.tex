\documentclass[a4j,uplatex]{jsarticle}
\usepackage{amsmath}

\title{確率情報理論第2回 解答}
\author{加藤まる}
\date{2020/03/02}

\begin{document}
\maketitle
\bf キーワード:期待値,分布関数,(同時確率),(条件付き期待値)
\\
\rm 確率論と情報理論第2回の内容です。

\section*{本日の問題}
1〜5の数字が書かれたカードがある。それぞれのカードを引く確率が、書かれた数字に比例する場合を考える。$(\Omega=\{ 1,2,3,4,5\})$
\begin{itemize}
  \item[(1)] 確率関数$P(X=k)$を求めよ。
  \begin{equation}
      1+2+3+4+5=15
  \end{equation} 
  確率は書かれた数字に比例することから、
  \begin{equation}
    P(X=k)=\frac{k}{15}
  \end{equation}
  \item[(2)] $E[X]$を求めよ。
  \begin{equation}
    \begin{split}
      E[X] &= {1}\times \frac{1}{15}+{2}\times \frac{2}{15}+{3}\times \frac{3}{15}+{4}\times \frac{4}{15}+{5}\times \frac{5}{15} \\
      &=\frac{55}{15}
    \end{split}
  \end{equation} 
  \item[(3)] 確率変数$X(\omega)=\omega$の分布関数$F(X)$を求めて図示せよ。
  \begin{equation}
    F(X)=
    \begin{cases}
      {0~~~X<1}\\
      {\frac{1}{15}~~~1\le X<2}\\
      {\frac{3}{15}~~~2\le X<3}\\
      {\frac{6}{15}~~~3\le X<4}\\
      {\frac{10}{15}~~~4\le X<5}\\
      {\frac{15}{15}~~~5\le X}\\
    \end{cases}
  \end{equation} 
\end{itemize}

\section*{おかわり問題}
裏と表が描かれてるコインがある。表の出る事象を1、裏の出る事象を0とする$(確率変数Y=\{ 0,1\})$。このとき、上の問題の確率変数$X$との同時確率を考えていく。
\begin{itemize}
  \item[(1)] $X$と$Y$の同時確率を求めよ。
  \\ 
  表作るの面倒なので我慢してください。
  \\
  \begin{equation}
    \begin{split}
      &P(X=1,Y=0)=P(X=1,Y=1)=\frac{1}{30}\\
      &P(X=2,Y=0)=P(X=2,Y=1)=\frac{2}{30}\\
      &P(X=3,Y=0)=P(X=3,Y=1)=\frac{3}{30}\\
      &P(X=4,Y=0)=P(X=4,Y=1)=\frac{4}{30}\\
      &P(X=5,Y=0)=P(X=5,Y=1)=\frac{5}{30}\\
    \end{split}
  \end{equation}
  \item[(2)] 条件付き期待値$E[X=k|Y=1]$を求めよ。
  \begin{equation}
    \begin{split}
      E(X=k|Y=1)&=\sum_{k=1}^5 k(P(X=k|Y=1))\\
      &=\sum_{k=1}^5 k\times \frac{P(X=k,Y=1)}{P(Y=1)}\\
      &=\frac{110}{30}
    \end{split}
  \end{equation} 
\end{itemize}

\end{document}