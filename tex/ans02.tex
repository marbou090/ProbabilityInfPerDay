\documentclass[a4j,uplatex,dvipdfmx]{jsarticle}
\usepackage{amsmath}
\usepackage{tcolorbox}

\title{確率情報理論第2回 解答}
\author{加藤まる}
\date{2020/03/02}

\begin{document}
\maketitle
\bf 
キーワード:確率関数,期待値\par
問題出典:東京図書発行 藤田岳彦著 「弱点克服 大学生の確率・統計」
\rm

\section*{本日の問題解答}
$f_X(x)=2e^{-2x}~~~(0<x\le 3),~ce^{-4x}~~~(x>3),~0~~~(Otherwise)$のとき、
\begin{itemize}
  \item[(1)] 定数cを求めよ。
   \begin{equation}
     \begin{split}
       1 =& \int_{0}^{3}e^{-2x}dx + \int_{3}^{\infty}ce^{-4x}dx\\
       & = \left[ -e^{-2x} \right]_0^3 + \left[ c \frac{e^{-4x}}{-4} \right]_3^\infty \\
       & = 1-e^{-6} + c\frac{e^{-12}}{4}
     \end{split}
   \end{equation}
   よって、$c=4e^6$
  \item[(2)] E[X]を求めよ。
  \begin{equation}
    \begin{split}
      E[X] &= \int_{0}^{3} x2e^{-2x} dx + c\int_{3}^{\infty}xe^{-4x}dx \\
      &= \int_{0}^{6} ue^{-u}\frac{du}{2} + c\int_{12}^{\infty}\frac{u}{4}e^{-u}\frac{du}{4} \\
      &= \frac{1}{2}\left[ -(u+1)e^{-u}  \right]_0^6 + \frac{c}{16}\left[ -(u+1)e^{-u} \right]_{12}^\infty \\
      &= \frac{1-7e^{-6}}{2} + \frac{13e^{-6}}{4} \\
      &= \frac{2-e^{-6}}{4}
    \end{split}
  \end{equation} 
\end{itemize}



\section*{本日の問題解説}
\begin{itemize}
  \item[(1)] 場合わけに注意しながら、全範囲で1になることから積分して求める問題。
  \item[(2)] (1)と同じく、場合わけに注意して期待値を計算する。\\ 
  解答ではuと置いているが、そのままでも部分積分で簡単に求められる。
\end{itemize}

\end{document}