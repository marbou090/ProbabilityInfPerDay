\documentclass[a4j,uplatex,dvipdfmx]{jsarticle}
\usepackage{amsmath}
\usepackage{tcolorbox}
\tcbuselibrary{breakable, skins, theorems}

\title{確率情報理論第17回}
\author{加藤まる}
\date{2020/03/17}

\begin{document}
\maketitle

\section*{本日の問題}

ある島には非常に珍しい鳥が生息している。研究員がその鳥の数(羽)を1年間に10回調査したところ、
平均25、不偏分散9であった。この結果から、この島には21を超える数の鳥が生息していると言えるかどうか検定せよ。
なお、有意水準は$α=0.05$とする。


\section*{おかわり問題}
ある貝の取引で、国内産表示のものが外国産なのではないかと疑っている。国内産の貝の大きさは
$N(6,\sigma ^2)$に従い、外国産の貝の大きさの母平均は国内産より小さいことがわかっている。
まず、10個の標本をとったところ、\\
1.2, 3.5, 6.3, 5.5, 3.6, 6.8, 4.2, 2.5, 7.0, 3.2(cm)\\
のデータが得られた。
\begin{itemize}
  \item[(1)]母分散が$\sigma=0.64$とわかっているとき、有意水準$z=0.05=5\%$で、帰無仮説$H_0 : \mu=6$で、
  対立仮説$H_1 : \mu <6$として、片側検定を実行せよ。
  \item[(2)]有意水準$z=5\%$のときの棄却域$\{\bar {X}<c_0 \}$を求めよ。\\
\end{itemize}
解答を加藤まる(まるぼう)にDiscordDMに送ると添削します(添削不要の場合DMは不要)。解答は夜にDiscordに貼るので自己採点してみてください。

\end{document}