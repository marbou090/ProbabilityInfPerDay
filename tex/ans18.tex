\documentclass[a4j,uplatex,dvipdfmx]{jsarticle}
\usepackage{amsmath}
\usepackage{tcolorbox}
\tcbuselibrary{breakable, skins, theorems}

\title{確率情報理論第18回 解答}
\author{加藤まる}
\date{2020/03/18}

\begin{document}
\maketitle
\bf キーワード:
\rm

\section*{本日の問題解答}
平均情報量(エントロピー)$H(P)=-\sum_{A\in \Omega} P(A)\log{P(A)}$である。
\begin{itemize}
  \item[(1)]公平なサイコロを振って出た目を表す確率変数Xの平均情報エントロピーを求めよ。\\
   公平なサイコロでは1から6までの目が当確率1/6で生じるので、
   \begin{equation}
     A=\{ a_1=1,a_2=2,\cdots,a_6=6 \}
   \end{equation}
   \begin{equation}
     p_i=\frac{1}{6}~~~k=1,2,\cdots,6
   \end{equation}
   となり、
   \begin{equation}
     H(X)=-\sum_{k=1}^6 p_i \log{p_i} =\log{6}\approx 0.77815
   \end{equation}

  \item[(2)]$p_1=p_6=\frac{1}{3}$でそれ以外は公平にでるサイコロがある。このサイコロを振って出た目を表す確率変数Yの平均情報エントロピーを求めよ。\\
  このサイコロでは、
  \begin{equation}
    p_1=p_6=\frac{1}{3},~~~p_2=p_3=p_4=p_5=\frac{1}{12}
  \end{equation} 
  となっている。よって、
  \begin{equation}
    \begin{split}
      H(Y)&=-\sum_{k=1}^6 p_i \log{p_k}\\
      &=-1\times \frac{1}{3}\log{\frac{1}{3}}-4\times \frac{1}{12}\log{\frac{1}{12}}\\
      &\approx 2.252
    \end{split}
  \end{equation}
  となる。
\end{itemize}


\section*{本日の問題解説}
情報量(エントロピー)とはその事象が起こったときに得られる情報の量のこと。確率pで起こる事象の情報量は、$-\log{p}$で表される。
(1)(2)より出る目に偏りのあるサイコロの方が、公平なサイコロよりもエントロピーが小さくなることがわかる。

\section*{おかわり問題解答}
同時エントロピー$\displaystyle H(X,Y)=-\sum_{x\in A} \sum_{x\in B}P(x,y)\log{P(x,y)}$である。\\
公平なコインを2回振ることを考える。
\begin{itemize}
  \item[(1)]最初の結果を表す確率変数をX,2回目の結果を表す確率変数をYとする。この時の同時エントロピーH(X,Y)を求めよ。
   このとき、(X,Y)の標本空間は
   \begin{center}
     [表,裏]$\times$ [表,裏]={(表,表),(表,裏),(裏,表),(裏,裏)}
   \end{center}
   である。これらの事象はすべて等確率1/4で生じる。したがって、
   \begin{equation}
     H(X,Y)=-4\times \frac{1}{4}log{\frac{1}{4}}=2\log{2}
   \end{equation}
   となる、
  \item[(2)]表が出た回数をZ,裏が出た回数をUとする。このとき、同時エントロピーH(Z,U)を求めよ。\\
  表と裏の出る回数はそれぞれ0回から2回までなので,(Z,U)の標本空間は
  \begin{center}
    {0,1,2}$\times${0,1,2}={(0,0),(0,1),(0,2),(1,0),(1,1),(1,2),(2,0),(2,1),(2,2)}
  \end{center} 
  となる。ここで、表がでた回数と裏が出た回数の和は2でなければならない。これらのことから、(Z,U)の事象の確率分布は、
  \begin{center}
    P(0,2)=(裏,裏)が出る確率=$\frac{1}{4}$\\
    P(2,0)=(表,表)が出る確率=$\frac{1}{4}$\\
    P(1,1)=(表,裏)あるいは(裏,表)が出る確率=$\frac{1}{2}$\\
    P(0,0)=P(0,1)=P(1,0)=P(1,2)=P(2,1)=P(2,2)=0
  \end{center}
  となる。したがって、
  \begin{equation}
    H(Z,U)=-2\times \frac{1}{4}\log{\frac{1}{4}}-\frac{1}{2}\log{\frac{1}{2}}=0.4515
  \end{equation}
  である。
\end{itemize}
\section*{おかわり問題解説}
確率論での期待値や同時確率などの情報量版と考えるとよいと思う。本日の問題解説にある通り、情報量の定義をイメージしながら
考えるとよい。

\end{document}