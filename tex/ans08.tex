\documentclass[a4j,uplatex,dvipdfmx]{jsarticle}
\usepackage{amsmath}
\usepackage{tcolorbox}
\tcbuselibrary{breakable, skins, theorems}

\title{確率情報理論第8回 解答}
\author{加藤まる}
\date{2020/03/08}

\begin{document}
\maketitle
\bf キーワード:
\rm

\section*{本日の問題解答}
\begin{tcolorbox}[
  title = マルコフの不等式,
  ]
    確率変数$X$について、$P\{ X>0 \}=1$であるとき、期待値$\mu$を持つならば任意の正数$\epsilon$に対して
    \begin{center}
      $P\{ X\ge \epsilon \} \le \frac{\mu}{\epsilon}$
    \end{center}
  \end{tcolorbox}
確率変数$X$が指数分布$Ex(1)$に従うとき、確率$P(X\ge 2)$について、真の値とマルコフの不等式による上からの評価値を比較せよ。(値はかなり異なる)
真の値は $e^{-2} = 0.1353$である。\\
\begin{equation}
  \begin{split}
    E[X] &= \frac{1}{\lambda}\\
    &= 1
  \end{split}
\end{equation}
であるから、マルコフの不等式より
\begin{equation}
  \begin{split}
    P\{ X\ge \epsilon \} \le \frac{\mu}{\epsilon} \\ 
    P\{ X\ge 2 \} \le \frac{1}{2}
  \end{split}
\end{equation}
よって評価値は0.5である。

\section*{本日の問題解説}
マルコフの不等式はチェビシェフの不等式の証明に用いられる。

\newpage
\section*{おかわり問題解答}
\begin{tcolorbox}[
  title = イェンセンの不等式,
  ]
      $g$が2回微分可能で下に凸($g''(x)\ge 0$)であるなら、
      \begin{center}
        $g(E[X])\le E(g(X))$
      \end{center}
  \end{tcolorbox}
$a_1,a_2,\cdots ,a_n \ge 0$として、イェンセンの不等式を用いて$\displaystyle \frac{a_1+a_2+\cdots +a_n}{n}\ge \sqrt[n]{a_1a_2\cdots a_3}$\\を示せ。

$\displaystyle P(X=a_i)=\frac{1}{n}(i=1,2,\cdots,n)$とし、$g(x)=-\log{x}$とする。イェンセン不等式より、
\begin{equation}
  \begin{split}
    g(E[X]) &= -\log{ \left( \frac{a_1,a_2,\cdots , a_n}{n} \right)} \\
    & \le E[g(X)] \\
    &= -\frac{\log{a_1}+\log{a_2}+\cdots + \log{a_n}}{n}\\
    &= -\log(\sqrt[n]{a_1 a_2 \cdots a_n})
  \end{split}
\end{equation}
よって、$\displaystyle \sqrt[n]{a_1 a_2 \cdots a_n} \le \frac{a_1 + a_2 + \cdots a_n}{n}$である。\\
(等号は$a_1=a_2=\cdots = a_n$)

\end{document}