\documentclass[a4j,uplatex,dvipdfmx]{jsarticle}
\usepackage{amsmath}
\usepackage{tcolorbox}
\usepackage{amssymb}
\tcbuselibrary{breakable, skins, theorems}

\title{確率情報理論第17回 解答}
\author{加藤まる}
\date{2020/03/17}

\begin{document}
\maketitle
\bf キーワード:
\rm

\section*{本日の問題解答}
本日の問題出典:https://bellcurve.jp/statistics/course/10004.html \\ \\

ある島には非常に珍しい鳥が生息している。研究員がその鳥の数(羽)を1年間に10回調査したところ、
平均25、不偏分散9であった。この結果から、この島には21を超える数の鳥が生息していると言えるかどうか検定せよ。
なお、有意水準は$α=0.05$とする。
\\
帰無仮説$H_0$を「生息数は平均21である」とする。対立仮説$H_1$を「生息数は平均21を超える」として、片側t検定を行う。
\begin{equation}
  t=\frac{\bar{x}-\mu}{\sqrt{\frac{s^2}{n}}}=\frac{25-21}{\sqrt{\frac{3^2}{10}}} \fallingdotseq 4.22
\end{equation}
自由度9のt分布の片側5\% 点1.833である。したがって$1.833\le t$が棄却域となる。この時、$1.833\le 4.22$であるから、
帰無仮説を棄却。対立仮説を採択し、この鳥の数は生息数は21を超える。

\section*{本日の問題解説}
不偏分散9より、自由度9のt分布であることに注意する。

\section*{おかわり問題解答}
ある貝の取引で、国内産表示のものが外国産なのではないかと疑っている。国内産の貝の大きさは
$N(6,\sigma ^2)$に従い、外国産の貝の大きさの母平均は国内産より小さいことがわかっている。
まず、10個の標本をとったところ、\\
1.2, 3.5, 6.3, 5.5, 3.6, 6.8, 4.2, 2.5, 7.0, 3.2(cm)\\
のデータが得られた。
\begin{itemize}
  \item[(1)]母分散が$\sigma=0.64$とわかっているとき、有意水準$z=0.05=5\%$で、帰無仮説$H_0 : \mu=6$で、
  対立仮説$H_1 : \mu <6$として、片側検定を実行せよ。
  10個の標本データから、
  \begin{equation}
    \bar{x}=4.38,~~~\sigma^2=0.64
  \end{equation}
  より、
  \begin{equation}
    \frac{\bar{x}-6}{\sqrt{\frac{0.64}{10}}}=-6.4036<-1.645=-\Phi ^{-1}(0.05)=-u(0.05)
  \end{equation}
  となるので、帰無仮説$H_0$は棄却され、対立仮説$H_1$を採択する。つまり、取引した貝は外国産である。

  \item[(2)]有意水準$z=5\%$のときの棄却域$\{\bar {X}<c_0 \}$を求めよ。\\
  棄却域は
  \begin{equation}
    \frac{c_0-6}{\sqrt{\frac{0.64}{10}}}=-1.645
  \end{equation} 
  より、$c_0=5.584$である。よって、求める棄却域は$\{ \bar{X}<5.584 \}$である。
\end{itemize}

\section*{おかわり問題解説}
(2)では標準化した値での棄却域でなく、標本平均についての棄却域を考えていることに注意。

\end{document}