\documentclass[a4j,uplatex,dvipdfmx]{jsarticle}
\usepackage{amsmath}
\usepackage{tcolorbox}
\tcbuselibrary{breakable, skins, theorems}

\title{確率情報理論第6回 解答}
\author{加藤まる}
\date{2020/03/06}

\begin{document}
\maketitle
\bf キーワード:
\rm

\section*{本日の問題解答}
全ての目が均等な確率で出る(つまり一様分布に従う)サイコロを500回転がすことを考える。\\
中心極限定理を用いて、サイコロの出目の平均が近似的に従う正規分布を求めよ。\\
ランダムに生成された実数をそれぞれ $X_1, X_2,\cdots, X_500$ とすると, これらは互いに独
立に同一の連続一様分布 $U(1, 6)$ に従う. このとき一様分布 $U(1, 6)$ の期待値と分
散は
\begin{equation}
  E(X_i) = \frac{1+6}{2} = \frac{7}{2} \ \ \ V(X_i) = \frac{6^2-1}{12} = \frac{35}{12}
\end{equation}
このとき標本平均$\displaystyle \overline{X} = \frac{1}{500} \sum_{i=1}^{500} X_i$の期待値と分散は, 互いに独立であることから
\begin{equation}
  E(\overline{X}) = \frac{7}{2} \ \ \ V(\overline{X}) = \frac{\frac{35}{12}}{500}
\end{equation}
したがって中心極限定理により, $\overline{X}$ は正規分布$\displaystyle N \left( \frac{7}{2},{\frac{7}{1200}}\right)$に近似的に従う.


\section*{おかわり問題解説}
出題ミス

\end{document}