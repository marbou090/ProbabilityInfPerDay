\documentclass[a4j,uplatex,dvipdfmx]{jsarticle}
\usepackage{amsmath}
\usepackage{tcolorbox}
\tcbuselibrary{breakable, skins, theorems}

\title{確率情報理論第18回}
\author{加藤まる}
\date{2020/03/18}

\begin{document}
\maketitle

\section*{本日の問題}
平均情報量(エントロピー)$H(P)=-\sum_{A\in \Omega} P(A)\log{P(A)}$である。
\begin{itemize}
  \item[(1)]公平なサイコロを振って出た目を表す確率変数Xの平均情報エントロピーを求めよ
  \item[(2)]$p_1=p_6=\frac{1}{3}$でそれ以外は公平にでるサイコロがある。このサイコロを振って出た目を表す確率変数Yの平均情報エントロピーを求めよ。
\end{itemize}



\section*{おかわり問題}
同時エントロピー$\displaystyle H(X,Y)=-\sum_{x\in A} \sum_{x\in B}P(x,y)\log{P(x,y)}$である。\\
公平なコインを2回振ることを考える。
\begin{itemize}
  \item[(1)]最初の結果を表す確率変数をX,2回目の結果を表す確率変数をYとする。この時の同時エントロピーH(X,Y)を求めよ。
  \item[(2)]表が出た回数をZ,裏が出た回数をUとする。このとき、同時エントロピーH(Z,U)を求めよ。\\
\end{itemize}
解答を加藤まる(まるぼう)にDiscordDMに送ると添削します(添削不要の場合DMは不要)。解答は夜にDiscordに貼るので自己採点してみてください。

\end{document}